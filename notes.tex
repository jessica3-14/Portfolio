\documentclass{article}
\usepackage{amsmath}
\usepackage{amssymb}
\usepackage{amsthm}
\newtheorem{definition}{Definition}[subsection]
\newenvironment{amatrix}[1]{%
  \left(\begin{array}{@{}*{#1}{c}|c@{}}
}{%
  \end{array}\right)
}

\begin{document}
\paragraph{Notes 8/31/2021}
What is a linear system of equations anyways -
an m X n system is a collection of m linear equations
in n variables of the form 
\[a_{11}x_1+a_{12}x_2...+a_{n1}x_n=b_1\]
\[a_{12}x_1+a_{22}x_2...+a_{n2}x_n=b_2\]
\[a_{1m}x_1+a_{m2}x_2...+a_{nm}x_n=b_n\]
Xs are the variables, as are the coefficient matrix, coefficient matrix and the collection of b values determine the system
\hfill \break
First interesting case is the 2x2 matrix
\[x_1+x_2=2\]
\[x_1-x_2=2\]
Each equation corresponds to a line in the x1/x2 plane,
solution is a coordinate of their intersection: 
can be zero, one, or infinitely many solutions
\hfill \break
In this case we have the point $(2,0)$ as the single solution
\[x_1+x_2=2\]
\[x_1+x_2=1\]
Case of zero solutions - parallel and distinct b/c same slope and different x2 intercepts (inconsistent)
\[x_1+x_2=2\]
\[-x_1-x_2=-2\]
Case of infinitely many solutions - same line

\paragraph{Equivalent systems}
Let $S$ = the set of all vectors $(x_1...x_n)$ in $\mathbb{R}^n$ such that the vector is a solution to the general linear system-
it is a function of the linear system
\hfill \break
Cardinality can be 1 (ie unique solution),
cardinality 0 (inconsistent or no solution),
cardinality infinite is infinitely many solutions
\hfill \break
Linear systems with same solution set are equivalent
\hfill \break
Select two distinct equations from the system - ways to preserve equivalence (elementary row operations):
\hfill \break
1. Change order of equations
\hfill \break
2. Multiply equation by a scalar
\hfill \break
3. Add equation to another equation (can combine with scalar multiplication) ****PROVE THIS AS AN EXERCISE****

\paragraph{Square and triangular systems}
System is strict triangular form iff the k-th equation, coefficient of first k-1 variables are 0,
and coefficient of $x_k$ is nonzero,
for all $k=1,...,n$
\hfill \break
Benefit of these systems is ability to backsubstitute to solve with ease
\[\begin{bmatrix}
    x_1 & 2x_2 & x_3 & 3\\
    3x_1 & -1x_2 & -3x_3 & -1\\
    2x_1 & 3x_2 & x_3 & 4
\end{bmatrix}\]
pivot variable is first nonzero entry
\paragraph{9/2/21}
Strict triangular form not always possible, because may not be square? \\
\[\begin{amatrix}{5}
    1&1&1&1&1&1\\
    -1&-1&0&0&1&-1\\
    -2&-2&0&0&3&1\\
    0&0&1&1&3&-1\\
    1&1&2&2&4&1
\end{amatrix}\]
row operations: $R-1+R_2 \rightarrow R_2, 2R_1+R_3\rightarrow R_3, -R_1+R_5 \rightarrow R_5$ \\
step 2: $-2R_2+R_3\rightarrow R_3, -R_2 +R_4 \rightarrow R_4, -R_2+R_5 \rightarrow R_5$ \\
step 3: $-R_3+R_4 \rightarrow R_4, -R_3 +R_5 \rightarrow R_5$ \\
\[\begin{amatrix}{5}
    1&1&1&1&1&1\\
    0&0&1&1&2&0\\
    0&0&0&0&1&3\\
    0&0&0&0&0&-4\\
    0&0&0&0&0&-3
\end{amatrix}\]
End result -- extremely inconsistent, if $b$ were changed so that bottom two rows end in zero the solution set $S$
would be a 3-space in 5-space \\
\[\begin{amatrix}{5}
    1&1&1&1&1&1\\
    0&0&1&1&2&0\\
    0&0&0&0&1&3\\
    0&0&0&0&0&0\\
    0&0&0&0&0&0
\end{amatrix}\]
Goal is to describe the solution set,
$x_1+x_2+x_3+x_4+x_5=1, x_3+x_4+2x_5=0, x_5=3$ \\
S is is the set of vectors in $\mathbb{R}^5$ such that they solve the linear system \\
We have pivots in the columns corresponding to $x_1,x_3,x_5$ and these should be used as they are lead variables.
All other variables ($x_2,x_4$ in this case) are free variables. \\
Rewrite to solve for lead variable in terms of free --
$x_5=3, x_3+2x_5=-x_4, x_1+x_3+x_5=1-x_2-x_4$,
this leaves us with a 3x3 strictly traingular system, any choice of $x_2$ and $x_4$
uniquely determines a solution to the system. \\
With backsolving we find that $x_1=-x_2+4, x_3=-x_4-6,x_5=3, x_2 and x_4 \in \mathbb{R}$\\
\paragraph{Linear independence}
There are no solutions $(\alpha, \beta)$ such that $\alpha \bar{x} + \beta \bar{y} = \bar{0}$ \\
Equivalent to saying they're not scalar multiples of each other. \\
All orthogonal vectors are linearly independent but not all linearly independent vectors are orthogonal 
\paragraph{Row-echelon form}
\begin{definition}
    A matrix $A$ is in row-echelon form (REF) iff first nonzero entry in each row is 1, \\
    if row k has a nonzero entry, number of leading zero entries in k+1 greater than number in row k, \\
    All rows consisting entirely of zeros are below any row with a nonzero entry 
\end{definition}
RREF (reduced row-echelon form) is addition steps to zero out anything to the right of the pivots,
automatically expressing lead variables in terms of free variables\\
It is easiest to go bottom up and right to left when solving\\
Columns with free variables do not have restriction on number of nonzero entries they can have\\
For reduced form, additional condition to the definition is that the first nonzero entry in each row is the only nonzero entry in its column

\paragraph{Homogeneous systems}
An $m \times n$ linear system of the form $(A \mid \bar{0})$ ie. the second bit is the zero vector.\\
An $m \times n$ homogeneous system has a nonzero solution if $n>m$\\
A homogeneous system is always consistent since the zero vector is always in the solution set \\
Let $U$ be the row-echelon form of the matrix $A$. \\
The matrix $U$ has at most $m$ nonzero rows because it has $m$ rows,
thus there are at most $m$ lead variable. \\
If $n>m$, then there must be at least one free variable, hence it has infinitely many nonzero solutions\\



\end{document}